\chapter{Introduction and hypothesis}

The introduction should establish the context of the work being reported and should answer the questions:

\begin{enumerate}
  \item Why and how is the work considered in this report relevant?
  \item What has been reported (and by who) within the field (state-of-the-art)?
  \item What is the purpose (hypothesis) of the work?
  \item What is the approach and possible outcome of the work?
\end{enumerate}

\noindent in short: \emph{''What am I studying? Why is it relevant? What is already know? How will the study advance our knowledge?''}\\

\section{Language and content}
You can write your thesis in English or Danish, thats your choice. It is important that you carefully read through your document asking yourself if what you are reading really is necessary for understanding the message or if the present material could be put in an appendix or an annex or perhaps it is not to be part of the thesis. You also need to ask yourself if every abbriviation or symbol is explained before it is used for the first time and if it is present in the nomenclature, which should be placed behind the appendices just before the references. An example:\\
\emph{\ldots glass can be delivered as both fully toughened- or fully tempered glass (FTG)\nomenclature[aftg]{FTG}{Fully Tempered/Toughened Glass} or heat-strengthened glass (HSG)\nomenclature[ahsg]{HSG}{Heat-Strengthened Glass} \ldots FTG is apparently stronger than HSG due to the residual stresses \ldots}\\

The following ''guidelines'' may assist you in determining what should go where in your thesis:

\subsection{Main text}
The main text should contain what is necessary to describe and answer what have been stated in the introduction without loosing sight. This includes how the main ideas have been solved including what assumptions you've made, sources of error, their significance etc. However stuff like experimental raw data, drawings for manufacturing the test setup, detailed derivations of analytical solutions (unless that's a main part of the thesis) for verifying numerical models etc. should not be in the main text. The main text should be kept as clean as possible without the reader loosing the principal ideas behind how the results were obtained and what assumptions have been made. It is also clear that repetition of information should not be present in the main document unless it has a clear purpose (e.g. preventing the reading from scrolling back and forth in the document all the time).

In short, the main text should be easy to read without too many detours. You have to keep a "red line" through your document.

\subsection{Appendix}
The appendix is also part of the main document but starts after the conclusions and contain valuable and more detailed information on what has been done. In the appendix long mathematical derivations, detailed numerical tests such as e.g. convergence and continuity. In principal all the details of the work described in the main document so that the reader is capable of repeating the work done by following the main text supported by the appendix.

\subsection{Annex}
The annex is a separate document or in this case a digital media such as a CD-ROM, DVD or USB stick where things such as experimental raw data should be located, pictures from testing, computer code, digital files, e.g. numerical models, CAD drawings etc., but also correspondence with e.g. companies or experts etc. Also a folder containing the litterature (as PDF's) can go into the annex.

\section{References}
This section is likely to contain many references to people who have done work in the same field. Without training it is often difficult to decide when a reference is needed and appropriate and how the citation should be done.

\subsection{Reference style}
There exist a number of different styles for citations and this is a matter of taste, however, a recommendation would be to use the \emph{Harvard referencing style} where the in-text citations includes authors and year and the publication list is given in alphabetic order. Stating the author names and years in the text makes it much more readable since you don't have to look up references you know. Furthermore, it is recommended to put the list of references on the last page after the appendix which makes it easy to find. More information on this can be found on the internet, e.g. \citep{hawardStyleOnline}.

\subsection{What to reference}
Citations should be used every time you make a claim that is not based on a well-known fact or common knowledge or if you quote or paraphrase somebody. The last part also goes for graphics; say you redraw a figure, then you need to cite the original reference, perhaps with a short note stating that you were inspired from that. Furthermore, if you do not cite the original work this should also be stated. More information can be found on the web, e.g.  \citep{PrincestonWhenToCite}.

Remember that the source is also important and for scientific work books articles and proceedings are still more trustworthy resources than the internet and should therefore be preferred.

