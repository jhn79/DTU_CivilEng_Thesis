\chapter{Project management}

\section{Time schedule}
One of the keys to success when doing such a project is planning and management of your time. You should therefore not underestimate the importance of having a time schedule for the whole project period. Such a time schedule should be ready from day one and (of course) it can be updated during the project. 


\section{Supervisor meetings}
It might differ how supervisors are guiding students, however, remember that your supervisor might be busy and in order to benefit most from meeting you should prepare your questions and perhaps even email them before meeting. Your prepared questions should be stated as clear as possible.

You might also be asked to give a presentation on the status of your project. This serves two purposes; the first one is to become comfortable with presenting your work and secondly it provides an update for the supervisors (and in case of cluster projects/project families, your colleagues).

